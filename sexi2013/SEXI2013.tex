% ----------------------------- %
% Paper for SEXI2013            %
% http://sexi2013.org/          %
% ----------------------------- %
% Full paper:     Nov 30, 2012  %
% Notification:   Dec 17, 2012  %
% Camera-ready:   Feb  5, 2013  %
% ----------------------------- %
% Maximum pages: 2              %
% ----------------------------- %

\documentclass[runningheads,a4paper]{llncs}
\usepackage[T1]{fontenc}
\usepackage[utf8]{inputenc}


% References
\usepackage[pdftex,urlcolor=black,colorlinks=true,linkcolor=black,citecolor=black]{hyperref}
\usepackage[capitalise,nameinlink]{cleveref}
\crefname{subsection}{Subsection}{Subsections}

% Better typography
\usepackage[activate=compatibility]{microtype}

% Todo macro
\usepackage{color}
\newcommand{\todo}[1]{\noindent\textcolor{red}{{\bf \{TODO} #1{\bf \}}}}

% Keywords command
\newcommand{\keywords}[1]{\par\addvspace\baselineskip
\noindent\keywordname\enspace\ignorespaces#1}

\begin{document}

\mainmatter

\title{Identifying Analogue Recording Artifacts\\
In the Age of Online Video Platforms}

\author{Thomas Steiner\inst{1} \and
        Seth van Hooland\inst{2} \and
        Ruben Verborgh\inst{3}}

\institute{
Universitat Polit\'ecnica de Catalunya -- Department {\scshape lsi},
  08034 Barcelona, Spain\\
  \urldef{\emails}\path|tsteiner@lsi.upc.edu|\emails\\
\and
Université Libre de Bruxelles\\ Information and Communication Science Department\\
     Avenue F.\ D.\ Roosevelt, 50 CP 123\\ B-1050 Brussels, Belgium\\
     \urldef{\emails}\path|svhooland@ulb.ac.be|\emails\\
\and
Ghent University -- {\scshape ibbt}, {\scshape elis} -- Multimedia Lab\\
  Gaston Crommenlaan 8 bus 201, B-9050 Ledeberg-Ghent, Belgium\\
  \urldef{\emails}\path|ruben.verborgh@ugent.be|\emails\\
}

\maketitle

% Ignore affiliation note numbers: start footnotes from the beginning.
\setcounter{footnote}{0}

\begin{abstract}
In this position paper, we describe how analogue recording artifacts
stemming from digitalized VHS tapes such as
grainy noises, ghosting, or synchronization problems
can be identified at Web-scale in order to
allow for filtering out content of low recording quality.
\end{abstract}

\keywords{Video Digitalization, VHS, Online Video}

\section{Introduction}

Online video is one of the fastest growing Internet industries.
With the latest statistics of a~well-known online video
platform\footnote{\url{http://www.youtube.com/t/press_statistics}}
of 72 uploaded hours of video per minute,
it becomes evident that efficient search, recommendation, and
navigation capabilities are required in order to use 
video platforms in a~meaningful way. 
Common online video platforms typically allow their users
(i)~to search for content based on full-text query terms
that are matched against textual descriptions
of the video like title or description,
or (ii)~to browse the archive of a~platform by category or channel,
usually based on video tags.
Users then retrieve a~top-$n$ ranked list of videos
that match a~given categorization
or query term, ranked by ranking criteria such as
\emph{relevancy}, \emph{view count},
\emph{user rating}, or \emph{upload date}.
The default ranking criteria normally being
\emph{relevancy}---a~platform-specific \emph{black box}
ranking criterium---advanced and frequently returning power-users
may prefer more transparent and traceable ranking criteria
such as the popularity-based \emph{view count}
and \emph{user rating}, or the stack-based
LIFO (last in, first out) ranking criterium \emph{upload date}.
In this position paper, we suggest a~computer vision-based
approach to a~problem with content uploaded a~long time ago
that occurs when users rank content
by the popularity-based ranking criterium \emph{view count}.

\section{Problem Statement}

\section{Conclusion}

\bibliographystyle{splncs03}
%\bibliography{references}

\end{document}
