% ----------------------------- %
% Paper for SEXI2013            %
% http://sexi2013.org/          %
% ----------------------------- %
% Full paper:     Nov 30, 2012  %
% Notification:   Dec 17, 2012  %
% Camera-ready:   Feb  5, 2013  %
% ----------------------------- %
% Maximum pages: 2              %
% ----------------------------- %

\documentclass[runningheads,a4paper]{llncs}
\usepackage[T1]{fontenc}
\usepackage[utf8]{inputenc}


% References
\usepackage[pdftex,urlcolor=black,colorlinks=true,linkcolor=black,citecolor=black]{hyperref}
\usepackage[capitalise,nameinlink]{cleveref}
\crefname{subsection}{Subsection}{Subsections}

\usepackage{graphicx}
\usepackage{caption}
\usepackage{subcaption}

% Better typography
\usepackage[activate=compatibility]{microtype}

% Todo macro
\usepackage{color}
\newcommand{\todo}[1]{\noindent\textcolor{red}{{\bf \{TODO} #1{\bf \}}}}

% Keywords command
\newcommand{\keywords}[1]{\par\addvspace\baselineskip
\noindent\keywordname\enspace\ignorespaces#1}

\begin{document}

\mainmatter

\title{Identifying {\scshape vhs} Recording Artifacts\\
in the Age of Online Video Platforms}

\titlerunning{Identifying {\scshape vhs} Recording Artifacts}
\authorrunning{Identifying {\scshape vhs} Recording Artifacts}

\author{Thomas Steiner\inst{1} \and
        Seth van Hooland\inst{2} \and
        Ruben Verborgh\inst{3}}

\institute{
Universitat Politècnica de Catalunya -- Department {\scshape lsi}\\
  \urldef{\emails}\path|tsteiner@lsi.upc.edu|\emails\\
\and
Université Libre de Bruxelles -- Information and Communication Science Dept.\\
     \urldef{\emails}\path|svhooland@ulb.ac.be|\emails\\
\and
Ghent University -- iMinds -- Multimedia Lab\\
  \urldef{\emails}\path|ruben.verborgh@ugent.be|\emails\\
}

\maketitle

% Ignore affiliation note numbers: start footnotes from the beginning.
\setcounter{footnote}{0}

\begin{abstract}
In this position paper, we describe how analogue recording artifacts
stemming from amateurishly digitalized {\scshape vhs} tapes like
grainy noises, ghosting, or synchronization issues
can be identified at Web-scale via crowdsourcing
in order to filter out content of low recording quality.
\end{abstract}

\keywords{Video Digitalization, Video Digitation, Video Home System~({\scshape vhs})}

\section{Introduction}

Online video is one of the fastest growing Internet industries.
With the latest statistics of a~well-known online video
platform\footnote{\url{http://www.youtube.com/t/press_statistics}}
of 72 uploaded hours of video per minute,
it becomes evident that efficient search, recommendation, and
navigation capabilities are required in order to use 
video platforms in a~meaningful way. 
Common online video platforms typically allow their users
(i)~to search for content based on full-text query terms
that are matched against textual descriptions
of the video like its title or description,
or (ii)~to browse the archive of a~platform by category or channel,
usually based on video tags.
Users are presented a~top-$n$ ranked list of videos
that match a~given category
or query term, ranked by ranking criteria such as
\emph{relevancy}, \emph{view count},
\emph{user rating}, or \emph{upload date}.
The default ranking criteria normally being
\emph{relevancy}---a~platform-specific \emph{black box}
ranking criterium---advanced and frequently returning power-users
may prefer more transparent and traceable ranking criteria
such as the popularity-based \emph{view count}
and \emph{user rating}, or the stack-based
{\scshape lifo} (last in, first out) ranking criterium \emph{upload date}.

In this position paper, we suggest a~computer vision-based
approach to a~problem with digitalized {\scshape vhs} content.
Typically, such videos were uploaded in the early to mid 2000s
and thus have accumulated many views,
however, due to the oftentimes amateurish digitalization process,
are of inferior quality and greatly limit the viewing experience,
albeit the poster preview of such videos may seem acceptable.
Digitalized {\scshape vhs} videos often appear prominently
when users rank content by the popularity-based
ranking criterium \emph{view count}.

\section{Problem Statement}

Video denoising is the process of removing noise from a~video signal.
Typical issues with digitalized {\scshape vhs} videos include 
ghosting (\autoref{fig:ghosting}),
color-specific degradation,
brightness and color channel interferences,
chaotic line shift at the end of frames,
and wide horizontal noise strips (\autoref{fig:distortion}).
Such distortions can seriously limit the viewing experience
on video platforms.
In consequence, we propose a~scalable, crowdsourced way
to identify poorly digitalized videos.

\vspace{-1em}

\begin{figure}
  \centering
  \begin{subfigure}[b]{0.45\linewidth}
    \centering
    \includegraphics[width=\textwidth]{ghosting.png} 
    \caption{Ghosting
    [\url{http://goo.gl/NXJzw}]}
     \label{fig:ghosting}
  \end{subfigure}
  \begin{subfigure}[b]{0.45\linewidth}
    \centering
    \includegraphics[width=\textwidth]{distortion.png} 
    \caption{Distortion
    [\url{http://goo.gl/zRiEQ}]}
     \label{fig:distortion}
  \end{subfigure}  
  \label{fig:artifacts}
  \caption{Typical {\scshape vhs} artifacts after amateurish digitalization.}
\end{figure}

\vspace{-3em}

\section{Approach}

In~\cite{steiner2011crowdsourcing}, we have introduced
a~generic crowdsourcing framework for the automatic and scalable
annotation of HTML5 video:
while a~user watches a~video, the framework in the background
unobtrusively annotates it, \emph{e.g.}, as demonstrated
in the concrete case, to extract events.
The annotation framework being generic,
we can imagine a~video denoising algorithm
as presented by Yang in~\cite{yang2009videonoise}
being applied to a~video that is currently played
to detect if it suffers from {\scshape vhs} artifacts.
Over time, \emph{individual} users watching low quality videos
create enough signals to eventually down-rank them
\emph{globally} for quality issues despite their view count.

\vspace{-0.5em}

\section{Conclusion}

In this position paper, we have presented a~crowdsourced,
scalable approach to detect {\scshape vhs} digitalization artifacts,
where users by watching videos do useful work such as 
detecting {\scshape vhs} artifacts as a~by-product,
and thus over time allowing video platforms to rank down
low quality content by leveraging the crowd.

\vspace{-0.5em}

\footnotesize
\bibliographystyle{splncs03}
\bibliography{references}

\end{document}
