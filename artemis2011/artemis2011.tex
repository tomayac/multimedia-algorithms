\documentclass[10pt,twocolumn,letterpaper]{article}

\usepackage{iccv}
\usepackage{times}
\usepackage{epsfig}
\usepackage{graphicx}
\usepackage{amsmath}
\usepackage{amssymb}

% Include other packages here, before hyperref.

% If you comment hyperref and then uncomment it, you should delete
% egpaper.aux before re-running latex.  (Or just hit 'q' on the first latex
% run, let it finish, and you should be clear).
\usepackage[pagebackref=true,breaklinks=true,letterpaper=true,colorlinks,bookmarks=false]{hyperref}


% \iccvfinalcopy % *** Uncomment this line for the final submission

\def\iccvPaperID{****} % *** Enter the ICCV Paper ID here
\def\httilde{\mbox{\tt\raisebox{-.5ex}{\symbol{126}}}}

% Pages are numbered in submission mode, and unnumbered in camera-ready
\ificcvfinal\pagestyle{empty}\fi
\begin{document}

%%%%%%%%% TITLE
\title{}

\author{Thomas Steiner\\
Google Germany GmbH\\
ABC Str. 19\\
{\tt\small tomac@google.com}
% For a paper whose authors are all at the same institution,
% omit the following lines up until the closing ``}''.
% Additional authors and addresses can be added with ``\and'',
% just like the second author.
% To save space, use either the email address or home page, not both
\and
Ruben Verborgh\\
Ghent University -- IBBT, ELIS -- Multimedia Lab\\
Gaston Crommenlaan 8 bus 201, B-9050 Ledeberg-Ghent, Belgium
{\small ruben.verborgh@ugent.be}
}

\maketitle
% \thispagestyle{empty}

%%%%%%%%% ABSTRACT
\begin{abstract}
   Lorem ipsum
\end{abstract}

%%%%%%%%% BODY TEXT
\section{Introduction}



\begin{figure}[t]
\begin{center}
\fbox{\rule{0pt}{2in} \rule{0.9\linewidth}{0pt}}
   %\includegraphics[width=0.8\linewidth]{egfigure.eps}
\end{center}
   \caption{Example of caption.  It is set in Roman so that mathematics
   (always set in Roman: $B \sin A = A \sin B$) may be included without an
   ugly clash.}
\label{fig:long}
\label{fig:onecol}
\end{figure}


\begin{figure*}
\begin{center}
\fbox{\rule{0pt}{2in} \rule{.9\linewidth}{0pt}}
\end{center}
   \caption{Example of a short caption, which should be centered.}
\label{fig:short}
\end{figure*}


\section{Related work}
\subsection{Shot boundary detection}

Video fragments consist of shots, which are sequences of consecutive frames from a single viewpoint,
representing an continuous action in time and space.
The topic of shot boundary detection has already been described extensively in literature.
While some specific issues still remain (notably gradual transitions and false positives due to large movement or illumination changes),
the problem is considered resolved for many cases~\cite{Hanjalic2002, Yuan2007}.
Below, we present an overview of several well-known categories of techniques.

Pixel comparison methods~\cite{Hampapur1994, Zhang1993} construct a discontinuity metric based on differences in color or intensity values of corresponding pixels in successive frames.
This dependency on spatial location makes this technique very sensitive to (even global) motion.
Various improvements have been suggested, such as prefiltering frames~\cite{Zhang1995},
but pixel-by-pixel comparison methods proved inferior in the end and have steered research to other directions.

A related method is histogram analysis~\cite{Smeaton1999}, where changes in frame histograms are used to justify the boundaries of shots.
Their insensitivity to spatial information within a frame makes histograms less prone to partial and global movements in a scene.
We can argue as a drawback that even visually very dissimilar frames can have similar overall histograms.
For example, different shots in the same scene can be difficult to distinguish because of similar color information.

As a compromise, a third group of methods consisted of a trade-off between the above two techniques~\cite{Ahmed1999}.
Different histograms of several, non-overlapping blocks are calculated for each frame,
thereby categorizing different regions of the image with its own color-based, space-invariant fingerprint.
The results are promising while computational complexity is kept to a minimum,
which is why we have chosen a variation on this approach in this paper.

Other approaches to shot boundary detection include the comparison of mean and standard deviations of frame intensities~\cite{Lienhart1999}.
Detection using other features such as edges~\cite{Zabih1995} and motion~\cite{Bouthemy1997} have also been proposed.
However, Gargi \emph{et~al.}\ have shown that these more complex methods do not necessarily outperform histogram-based approaches~\cite{Gargi2000}.
A detailed overview and comparison can be found in Yuan \emph{et~al.}~\cite{Yuan2007}.

Another interesting track is shot boundary detection in the compressed domain~\cite{Yeo1995},
which is especially relevant for online video, as compression plays a major part herein.
Unfortunately, even though this article focuses on online video, client-side manipulation of the raw, compressed stream would introduce a lot of computational overhead and make our implementation codec-dependent.

{\small
\bibliographystyle{ieee}
\bibliography{artemis2011}
}


\end{document}
