% Template for ICME-2013 paper; to be used with:
%          spconf.sty  - ICASSP/ICIP LaTeX style file, and
%          IEEEbib.bst - IEEE bibliography style file.
% --------------------------------------------------------------------------
\documentclass{article}
\usepackage{spconf,amsmath,epsfig}

\usepackage[utf8]{inputenc}

\usepackage{subfig}

% autoref command
\usepackage[hyphens]{url}
\usepackage[pdftex,urlcolor=black,colorlinks=true,linkcolor=black,citecolor=black]{hyperref}
\def\sectionautorefname{Section}
\def\subsectionautorefname{Subsection}
\def\subfigureautorefname{Subfigure}

% listings and Verbatim environment
\usepackage{fancyvrb}
\usepackage{relsize}
\usepackage{listings}
\usepackage{verbatim}
\newcommand{\defaultlistingsize}{\fontsize{8pt}{9.5pt}}
\newcommand{\inlinelistingsize}{\fontsize{8pt}{11pt}}
\newcommand{\smalllistingsize}{\fontsize{7.5pt}{9.5pt}}
\newcommand{\listingsize}{\defaultlistingsize}
\RecustomVerbatimCommand{\Verb}{Verb}{fontsize=\inlinelistingsize}
\RecustomVerbatimEnvironment{Verbatim}{Verbatim}{fontsize=\defaultlistingsize}
\lstset{frame=lines,captionpos=b,numberbychapter=false,escapechar=§,
        aboveskip=2em,belowskip=1em,abovecaptionskip=0.5em,belowcaptionskip=0.5em,
        framexbottommargin=-1em,basicstyle=\ttfamily\listingsize\selectfont}

\pagestyle{empty}

\begin{document}\sloppy

% for 1st, 2nd, etc. superscripting
\newcommand{\ts}{\textsuperscript}

% Title.
% ------
\title{Tell me why! Ain't nothin' but a mistake?\\ Describing Media Item Differences with Media Fragments URI}
%
% Single address.
% ---------------
\name{Anonymous ICME submission}
\address{}

\maketitle

%
\begin{abstract}
We have developed a~tile-wise histogram-based
media item deduplication and clustering algorithm
with additional high-level semantic matching criteria.
In this paper, we investigate whether the addressing scheme
Media Fragments  {\sc uri} provides a~feasible and practicable way
to describe media item differences
between media items of type photo and/or video.
\end{abstract}
%
\begin{keywords}
Media Fragments {\sc uri}, Media Fragments, Media Items, Deduplication, Social Networks
\end{keywords}
%
\section{Introduction}
\label{sec:introduction}

The \emph{Backstreet Boys}~({\sc bsb}) are a~boy band
formed in~1993 in Orlando,~FL
that has sold over 130~million records worldwide,
making them the best-selling boy band of all time.
In~2013, the band will celebrate their 20\ts{th}~anniversary
with a~new album and a~world tour.
Reason enough for us to make them titular saint of this paper
with their hit song \emph{I~Want It That Way}
from the album \emph{Millennium}.
While the spike of their career was in the late 90s,
even today, people still actively share,%
\footnote{{\sc bsb} on social networks: \url{http://bit.ly/backstreet-gplus}
and \url{http://bit.ly/backstreet-fb},
both accessed 03/04/2013}
publish, and follow the group on \emph{social networks}.

\subsection{Previous Work}
\label{sec:previous-work}

Social networks are at the heart of our research on event summarization,
specifically deduplicating \emph{exact-} and \emph{near-duplicate}
media items that optionally accompany textual status messages
referred to as \emph{microposts} on multiple social networks. 
In the context of our research, we define a~\emph{media item}
as either a~photo (image) or video
that was \emph{publicly} shared or published
on at least one social network.
\autoref{fig:near-duplicate} shows an example
where two users of the social networks Facebook and Google+
independently of each other share a~\emph{near-duplicate} media item
in form of the music video \emph{Everybody}
performed by the \emph{Backstreet Boys}.
In order to detect and cluster such occurrences
of \emph{exact-} and \emph{near-duplicate}
media items being shared independently across social networks,
we have implemented a~tile-wise histogram-based algorithm
with additional high-level semantic matching criteria
that in previous work was shown to work effectively for several events.


\begin{figure}[b!]
  \centering
  \includegraphics[width=0.75\linewidth]{./backstreetboys.png}
  \caption{\emph{Near-duplicate} music video \emph{Everybody}
    by the \emph{Backstreet Boys} shared
    independently on Facebook and Google+}
  \label{fig:near-duplicate}
\end{figure}

\subsection{Motivation and Research Question}
\label{sec:motivation-and-research-question}

During previous experiments on clustering event-related media items,
we noticed that human raters wanted to know \emph{why}%
\footnote{Tell me why! Ain't nothin' but a mistake?}
certain media items were clustered as \emph{exact-} or \emph{near-duplicates}.
In consequence, in this paper, we investigate in how far
Media Fragments {\sc uri}~\cite{troncy2012mediafragments}
provides a~feasible and practicable way
to tell raters why media items were clustered.
As we deal with media items of type photo and/or video,
we make simultaneous use of two types of media fragment dimensions,
the temporal dimension and the spatial dimension.

\subsection{Paper Structure}
\label{sec:paper-structure}

The remainder of this paper is structured as follows.
In \autoref{sec:related-work}, we report on related work
on media fragments and digital storytelling.
In \autoref{sec:media-item-deduplication-algorithm},
we describe the tile-wise histogram-based
media item deduplication and clustering algorithm.


\section{Related Work}
\label{sec:related-work}

\noindent \textit{Media Fragments:}
Media Fragments {\sc uri}~\cite{troncy2012mediafragments} specifies
the syntax for constructing media fragment {\sc uri}s
and explains how to handle them
when used over the {\sc http} protocol~\cite{fielding1999http}.
The syntax is based on the specification of particular name-value pairs
that can be used in {\sc uri} fragment and {\sc uri} query requests
to restrict a~media resource to a~certain fragment.
Currently supported media fragment {\sc uri}s in the basic version
cover the temporal and the spatial dimension.
The temporal dimension denotes a~specific time range in the original media,
such as ``starting at second 10, continuing until second 20.''
The spatial dimension denotes a~specific range of pixels in the original media,
such as ``a~rectangle of size $ 100 \times 100 $
with its top-left at the coordinates $ (10, 10) $,''
where combinations of both dimensions are possible.

\noindent \textit{Digital Storytelling:}
Pizzi and Cavazza report in~\cite{pizzi2008debugging} on the development of
an authoring technology on top of an interactive storytelling system
that originated as a~debugging%
\footnote{Note, Pizzi and Cavazza use the term \emph{debugging} in the non-IT sense:
to check for redundancy, dead-ends, consistency, \emph{etc.} in authored stories}
tool for a~planning system.
Alexander and Levine define in~\cite{alexander2008storytelling}
the term \emph{Web~2.0 storytelling}, where people create \emph{microcontent}---%
small chunks of content, with each chunk conveying a primary idea or concept---%
that gets combined with social media to form coherent stories.
We use Media Fragments {\sc uri}s to help raters understand
the results of an algorithm by converting dry software debugging data
to digital stories.

\section{Media Item Deduplication Algorithm}
\label{sec:media-item-deduplication-algorithm}

Our near-duplicate media item clustering algorithm belongs to the family of
tile-wise histogram-based clustering algorithms.
As an additional semantic feature, the algorithm considers detected faces.
For two media items to be clustered,
the following conditions have to be fulfilled.

\begin{enumerate}
  \item Out of $m$ tiles of a~media item with $n$ tiles ($m \leq n$),
    at most $\textit{tiles\_threshold}$ tiles may differ not more than $\textit{similarity\_threshold}$
    from their counterpart~tiles.
  \item The numbers $f_1$ and $f_2$ of detected faces in both media items
    have to be the same.
    We note that we do not \emph{recognize} faces, but only \emph{detect} them.
\end{enumerate}

In order to illustrate the way the algorithm deduplicates media items,
\autoref{fig:algorithmdebug} shows a~debug view of the algorithm
for the two clustered media items related to the 
\emph{Backstreet Boys} music video.
Independent from the actual media item's aspect ratio,
the tile-wise comparison always happens based on a~potentially squeezed
square aspect ratio version.

\begin{figure}[b!]
  \centering
  \subfloat[From Facebook user]{
    \includegraphics[width=0.3\linewidth]{debug1.png}
  }                
  \subfloat[From Google+ user]{
    \includegraphics[width=0.3\linewidth]{debug2.png}
  }
  \caption{}
  \label{fig:algorithmdebug}  
\end{figure}


% References should be produced using the bibtex program from suitable
% BiBTeX files (here: strings, refs, manuals). The IEEEbib.bst bibliography
% style file from IEEE produces unsorted bibliography list.
% -------------------------------------------------------------------------
\bibliographystyle{IEEEbib}
\bibliography{icme2013template}

\end{document}
