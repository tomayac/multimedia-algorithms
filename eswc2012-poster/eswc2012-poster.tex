\documentclass[runningheads,a4paper,11pt]{llncs}
\usepackage{fullpage}

\usepackage[utf8]{inputenc}
\usepackage{graphicx}
\usepackage{amsmath}
\usepackage{amssymb}
\usepackage{subfig}

% proper encoding
\usepackage[T1]{fontenc}
% use Times
\usepackage{mathptmx}

% autoref command
\usepackage[pdftex,urlcolor=black,colorlinks=true,linkcolor=black,citecolor=black]{hyperref}
\def\sectionautorefname{Section}
\def\subsectionautorefname{Subsection}
\def\figureautorefname{Fig.}
\def\subfigureautorefname{Fig.}

% proper typography
\usepackage[protrusion,expansion,kerning,spacing,tracking]{microtype}

% URLs
\usepackage{url}

% todo macro
\usepackage{color}
\newcommand{\todo}[1]{\noindent\textcolor{red}{{\bf \{TODO} #1{\bf \}}}}

% listings and Verbatim environment
\usepackage{fancyvrb}
\usepackage{relsize}
\RecustomVerbatimCommand{\Verb}{Verb}{fontsize=\fontsize{9pt}{11pt}}
\RecustomVerbatimEnvironment{Verbatim}{Verbatim}{fontsize=\fontsize{9pt}{12pt}}
\usepackage{listings}
\lstset{frame=lines,captionpos=b,numberbychapter=false,escapechar=§,
        aboveskip=0em,belowskip=0em,abovecaptionskip=.3em,belowcaptionskip=0em,
        basicstyle=\ttfamily\fontsize{8.5pt}{9.5pt}\selectfont}

% linewrap symbol
\definecolor{grey}{RGB}{160,160,160}
\newcommand{\linewrap}{\raisebox{-.6ex}{\textcolor{grey}{$\hookleftarrow$}}}

% keywords command
\newcommand{\keywords}[1]{\par\addvspace\baselineskip
\noindent\keywordname\enspace\ignorespaces#1}

\begin{document}

\mainmatter

\title{Ranking Criteria for Media Items Illustrating Events\\ Stemming from Multiple Social Networks}

% use Courier from this point onward (used CM for author e-mails), and smaller in-text URIs
%\renewcommand{\ttdefault}{pcr}
%\renewcommand\UrlFont{\smaller\tt}

\author{Thomas Steiner\inst{1} \and Ruben Verborgh\inst{2} \and Rik Van de Walle\inst{2} \and Joaquim Gabarro\inst{1}}

\institute{Universitat Politècnica de Catalunya -- Department LSI\\
08034 Barcelona, Spain\\
\email{\{tsteiner, gabarro\}@lsi.upc.edu}
\and Ghent University -- IBBT, ELIS -- Multimedia Lab\\Gaston Crommenlaan 8 bus 201, B-9050 Ledeberg-Ghent, Belgium\\
\email{\{ruben.verborgh, rik.vandewalle\}@ugent.be}
}

\authorrunning{Ranking Criteria for Media Items Illustrating Events}
% (feature abused for this document to repeat the title also on left hand pages)
% a short form should be given in case it is too long for the running head
\titlerunning{Ranking Criteria for Media Items Illustrating Events}

\maketitle

%%%%%%%%% ABSTRACT
\begin{abstract}
Mobile devices such as smartphones or digital cameras together with social networks
enable people to create, share, and consume enormous amounts of media items
like videos or photos, both \textit{en route} or at home.
Such mobile devices---by pure definition---accompany their owners almost wherever they may go.
In consequence, mobile devices are omnipresent at all sorts of events
like keynote speeches at conferences, music concerts in stadiums,
or even natural catastrophes affecting whole areas or countries.
At such events---given a~stable network connection---media items are published
on social networks both as the event happens and afterwards.
Media items stemming from multiple social networks can serve to create authentic media galleries
that illustrate events from the view of event attendants and
that convey at least part of the atmosphere to those who did not attend the event.
In this paper, we present and discuss several ranking criteria that can help
triage and prune the potentially huge amount of available media items for a given event.
Besides obvious low-level \emph{visual} ranking criteria such as image or video resolution---and
in consequence leveraging the additional information coming from social networks---we introduce
\emph{textual} ranking criteria from associated microposts that accompany media items and
\emph{social} ranking criteria from social networking effects such as shares, mentions,
number of views, expressions of likes and dislikes, user diversity, etc.
Finally, we discuss media gallery specific \emph{aesthetic} ranking criteria such as
variety of the included media items and the balance of video \emph{vs.} image media items.
\end{abstract}

% back to normal size Computer Modern for URLs in bibliography
\renewcommand{\ttdefault}{cmvtt}
\renewcommand\UrlFont\tt

\bibliographystyle{splncs03}
\bibliography{eswc2012-poster}

\end{document}
