\documentclass[runningheads,a4paper,11pt]{llncs}
%\usepackage{fullpage}

\usepackage[utf8]{inputenc}
\usepackage{graphicx}
\usepackage{amsmath}
\usepackage{amssymb}
\usepackage{subfig}

% proper encoding
\usepackage[T1]{fontenc}
% use Times
\usepackage{mathptmx}

% autoref command
\usepackage[pdftex,urlcolor=black,colorlinks=true,linkcolor=black,citecolor=black]{hyperref}
\def\sectionautorefname{Section}
\def\subsectionautorefname{Subsection}
\def\figureautorefname{Fig.}
\def\subfigureautorefname{Fig.}

% proper typography
\usepackage[protrusion,expansion,kerning,spacing,tracking]{microtype}

% URLs
\usepackage{url}

% todo macro
\usepackage{color}
\newcommand{\todo}[1]{\noindent\textcolor{red}{{\bf \{TODO} #1{\bf \}}}}

% listings and Verbatim environment
\usepackage{fancyvrb}
\usepackage{relsize}
\RecustomVerbatimCommand{\Verb}{Verb}{fontsize=\fontsize{9pt}{11pt}}
\RecustomVerbatimEnvironment{Verbatim}{Verbatim}{fontsize=\fontsize{9pt}{12pt}}
\usepackage{listings}
\lstset{frame=lines,captionpos=b,numberbychapter=false,escapechar=§,
        aboveskip=0em,belowskip=0em,abovecaptionskip=.3em,belowcaptionskip=0em,
        basicstyle=\ttfamily\fontsize{8.5pt}{9.5pt}\selectfont}

% linewrap symbol
\definecolor{grey}{RGB}{160,160,160}
\newcommand{\linewrap}{\raisebox{-.6ex}{\textcolor{grey}{$\hookleftarrow$}}}

% keywords command
\newcommand{\keywords}[1]{\par\addvspace\baselineskip
\noindent\keywordname\enspace\ignorespaces#1}

\begin{document}

\mainmatter

\title{Ranking Criteria for Media Items Illustrating Events\\ and Stemming from Multiple Social Networks}

% use Courier from this point onward (used CM for author e-mails), and smaller in-text URIs
\renewcommand{\ttdefault}{pcr}
\renewcommand\UrlFont{\smaller\tt}

\author{Thomas Steiner\inst{1} \and Ruben Verborgh\inst{2} \and Rik Van de Walle\inst{2} \and Joaquim Gabarro\inst{1}}

\institute{Universitat Politècnica de Catalunya -- Department LSI, 08034 Barcelona, Spain\\
\email{\{tsteiner, gabarro\}@lsi.upc.edu}
\and Ghent University -- IBBT, ELIS -- Multimedia Lab, Ghent, Belgium\\
\email{\{ruben.verborgh, rik.vandewalle\}@ugent.be}
}

\authorrunning{Ranking Criteria for Media Items Illustrating Events}
% (feature abused for this document to repeat the title also on left hand pages)
% a short form should be given in case it is too long for the running head
\titlerunning{Ranking Criteria for Media Items Illustrating Events}

\maketitle

%%%%%%%%% ABSTRACT
\begin{abstract}
In this paper, we present, classify, and discuss ranking criteria that can help
triage and prune potentially huge amounts of media items for a~given event.
\end{abstract}

\section{Introduction}
Mobile devices like smartphones together with social networks
enable people to create, share, and consume media items
like videos or images.
They accompany their owners almost wherever they may go
and are omnipresent at all sorts of events.
Given a~stable network connection, event-related media items are published
on social networks both as events happen and afterwards.
Ranked media items stemming from multiple social networks
can serve to create authentic media galleries
that illustrate events and their atmosphere.
A~key feature for this task is the semantic enrichment of media items and the related metadata.

\section{Motivational Event Scenarios}
We assume (and are currently working on) functioning media item extractors
for multiple social networks that, given event-related search terms,
extract raw binary media items and associated metadata.
Events like music concerts, sports matches, or keynote speeches
are typically well covered social-media-wise.
The given contexts of known artists, sports teams, or speakers
allow for leveraging knowledge from the
Linked Open Data (LOD) cloud\footnote{Linked Open Data cloud: \url{http://lod-cloud.net/}}.
Concretely, a~video of the former German President \emph{Christian Wulff}
that gets shared on social networks
can be considered relevant for the---at time of writing---recent
event of his resignation speech,
if (i)~in the accompanying micropost the named entity represented by the DBpedia concept
\url{db:Resignation_speech} is detected,
and (ii)~his face is recognized in the video.
In the given example, this video might outrank a~video
of a journalist commenting on the case.
We note, however, that many more ranking criteria exist,
which we discuss and classify in the upcoming Section.

\section{Classification and Discussion of Ranking Criteria}
\textbf{Visual}
This category regards the actual contents of media items.
We distinguish \emph{low-level} and \emph{high-level} visual ranking criteria.
Examples for high-level ranking criteria are the detection of logos,
the recognition of faces, or the detection of camera shots.
Examples for low-level ranking criteria are file size, pixel resolution,
video duration, geolocation, date and time information, etc.

\noindent \textbf{Textual}
This category regards the microposts that accompany media items.
Typically, microposts provide a~description of media items.
Using named entity disambiguation tools,
textual content can be linked to LOD cloud concepts~\cite{Facebook2011}.

\noindent \textbf{Social}
This category regards social network effects like shares, mentions,
number of views, expressions of (dis)likes, user diversity, etc.
Previous work allows us to not only examine these effects
on a~\emph{single} social network (\emph{e.g.}, ReTweets on Twitter),
but in a~\emph{network-agnostic} way across multiple social networks.

\noindent \textbf{Aesthetic}
This category regards the desired final outcome after the ranking, \emph{i.e.},
the media gallery that illustrates an event and its atmosphere.
Studies exist for the aesthetics of
automatic photo book layout~\cite{Photo2011}
and photo aesthetics \emph{per se}~\cite{Photo2012}.
However, media gallery composition requires mixing videos and images.\\

\noindent We have run first media item ranking tests of combinations
of ranking criteria and ranking criteria in isolation.
While for some ranking criteria the preferences are straightforward
(\emph{e.g.}, higher pixel resolution is usually preferred over lower resolution),
for others, the preferences are not that clear (\emph{e.g.}, video duration).

\section{Conclusion and Future Work}
The main contribution of this paper is the groundwork of classifying ranking criteria
and their compilation in the first place.
Future work will focus on user studies in order to determine
weight factors of the ranking criteria.
Therefore, we will implement an interactive application,
where, via intuitive sliders, user study participants will be asked
to adjust the ranking criteria for a~set of different kinds of events
until they are happy with the final resulting ranked lists of media items.

% back to normal size Computer Modern for URLs in bibliography
\renewcommand{\ttdefault}{cmvtt}
\renewcommand\UrlFont\tt

\bibliographystyle{splncs03}
\bibliography{eswc2012-poster}

\end{document}
