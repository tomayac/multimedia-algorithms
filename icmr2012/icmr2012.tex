\documentclass{acm_proc_article-sp}

\usepackage[utf8]{inputenc}
\usepackage[T1]{fontenc}

\usepackage[activate=compatibility]{microtype}

% autoref command
\usepackage[pdftex,urlcolor=black,colorlinks=true,linkcolor=black,citecolor=black]{hyperref}
\def\sectionautorefname{Section}
\def\subsectionautorefname{Subsection}
\def\subfloatautorefname{Subfigure}

\usepackage[lofdepth,lotdepth]{subfig}

\usepackage{enumitem}

\usepackage{mathtools}

% give emph a normal fontsize
\let\oldemph\emph
\renewcommand{\emph}[1]{\oldemph{\fontsize{9}{9}\selectfont #1}}

% more readable footnote layout
\renewcommand{\footnotesize}{\fontsize{8pt}{10pt}}
\setlength{\footnotesep}{.5cm}

% todo macro
\usepackage{color}
\newcommand{\todo}[1]{\noindent\textcolor{red}{{\bf \{TODO}: #1{\bf \}}}}

% listings and Verbatim environment
\usepackage{fancyvrb}
\usepackage{relsize}
\usepackage{listings}
\usepackage{verbatim}
\newcommand{\defaultlistingsize}{\fontsize{8pt}{9.5pt}}
\newcommand{\inlinelistingsize}{\fontsize{8pt}{11pt}}
\newcommand{\smalllistingsize}{\fontsize{7.5pt}{9.5pt}}
\newcommand{\listingsize}{\defaultlistingsize}
\RecustomVerbatimCommand{\Verb}{Verb}{fontsize=\inlinelistingsize}
\RecustomVerbatimEnvironment{Verbatim}{Verbatim}{fontsize=\defaultlistingsize}
\lstset{frame=lines,captionpos=b,numberbychapter=false,escapechar=§,
        aboveskip=0.5em,belowskip=0em,abovecaptionskip=0em,belowcaptionskip=0em,
framexbottommargin=-1em,
        basicstyle=\ttfamily\listingsize\selectfont}

% use Courier from this point onward
\let\oldttdefault\ttdefault
\renewcommand{\ttdefault}{pcr}
\let\oldurl\url
\renewcommand{\url}[1]{\inlinelistingsize\oldurl{#1}}

% superscript for 1st, 2nd, etc.
%\newcommand{\superscript}[1]{\ensuremath{^{\textrm{#1}}}}
%\newcommand{\subscript}[1]{\ensuremath{_{\textrm{#1}}}}
%\newcommand{\th}[0]{\superscript{th}}
%\newcommand{\st}[0]{\superscript{st}}
%\newcommand{\nd}[0]{\superscript{nd}}
%\newcommand{\rd}[0]{\superscript{rd}}

% linewrap symbol
\definecolor{grey}{RGB}{130,130,130}
\newcommand{\linewrap}{\raisebox{-.6ex}{\textcolor{grey}{$\hookleftarrow$}}}

% more pleasing quote environment
\usepackage{tikz}
\newcommand*{\openquote}{\tikz[remember picture,overlay,xshift=-7pt,yshift=1pt]
     \node (OQ) {\fontfamily{fxl}\fontsize{16}{16}\selectfont``};\kern0pt}
\newcommand*{\closequote}{\tikz[remember picture,overlay,xshift=2pt,yshift=-4.5pt]
     \node (CQ) {\fontfamily{fxl}\fontsize{16}{16}\selectfont''};}
\renewenvironment{quote}%
{\setlength{\parindent}{1cm}\par\openquote}
{\closequote\vspace{-4.5pt}
}

\makeatletter
\let\@copyrightspace\relax
\makeatother

\begin{document}

\title{Getting the Bigger Picture -- Extracting Media Items Covering Events from Multiple Social Networks}

\numberofauthors{3}
\author{
\alignauthor
\textbf{Thomas Steiner}\\
	\affaddr{Univ. Polit\`{e}cnica de Catalunya}\\
	\affaddr{Department LSI}\\
	\affaddr{08034 Barcelona, Spain,}\\
	\affaddr{tsteiner@lsi.upc.edu}
\alignauthor
\textbf{Raphaël Troncy}\\
	\affaddr{EURECOM}\\
	\affaddr{06560 Sophia Antipolis}\\
	\affaddr{France}\\
	\affaddr{rtroncy@eurecom.fr}
\and
\alignauthor
\textbf{Ruben Verborgh}\\ 	
	\affaddr{Ghent University -- IBBT, ELIS}\\
	\affaddr{Multimedia Lab}\\
	\affaddr{9050 Ghent, Belgium}\\
	\affaddr{ruben.verborgh@ughent.be}
\alignauthor
\textbf{Joaquim Gabarró Vallés}\\
	\affaddr{Univ. Polit\`{e}cnica de Catalunya}\\
	\affaddr{Department LSI}\\
	\affaddr{08034 Barcelona, Spain,}\\
	\affaddr{gabarro@lsi.upc.edu}	
}

\maketitle

\begin{abstract}
Core contributions:
Social network and media hosting platform agnostic media item search, and search results alignment.
Search results semantic enrichment by putting microposts and media items in relation.
Cross-channel popularity analysis.
Visual clustering, photos contained in videos.
\end{abstract}

\category{H.3.4}{Information Systems}{Information Storage and Retrieval}[World Wide Web]
\category{H.3.5}{Online Information Services}{Web-based services}

\keywords{\todo{Keywords}}

\section{Introduction} \label{sec:introduction}

\subsection{Definitions}
\subsubsection{Media Item}

\subsubsection{Micropost}

\subsubsection{Event}

\section{Social Networks and Media Items}

\subsection{Social Networks vs. Hosting Platforms}
The boundary between social networks and media hosting platforms is fluid.
There are media hosting platforms where people can upload their content with optionally content viewers -- in a relation to the original uploader or not -- having the possibility to react in form of comments, or in form of likes or dislikes.
An example is YouTube.
There are social networks where people can update their status, post links to stories, or upload their content with necessarily viewers -- standing in a relation or not -- having the option to react.
An example is Facebook.
Finally, there are hybrid forms, where social networks -- typically via third party applications -- integrate with media hosting platforms.
An example is the TweetDeck for Twitter application and their integration with TwitPic.

\subsection{Media Item Extraction}
Talk about media item recall (ratio items that match the query vs. items that match the query with media items).

\subsection{API Access vs. Web Scraping}
An \emph{Application Programming Interface (API)} in the sense of Web-based API is a programmatic specification intended to be used as an interface by software components on client and server to communicate with each other.

\emph{Web scraping} is the process of automatically extracting information from websites.
Web scraping involves practical solutions based on existing technologies that are often entirely ad hoc.
Examples of such technologies are regular expressions or DOM parsing of Web pages into a DOM tree.

The difference to the somewhat related concept of \emph{screen scraping} is that screen scraping relies on the visual layout of a website, whereas Web scraping relies on the textual and/or hierarchical structure of websites.

Social networks today are very much perceived as ``walled gardens'', excellently illustrated by a cartoon by David Simonds~(\autoref{fig:DavidSimonds}).
As with Orwell, where some animals are more equal than others, some social networks are more walled than others.
Some social networks have full read and write access via specified APIs.
An example is Twitter.
Other social networks have read access via APIs.
An example is Google+.
Interestingly, some media hosting platforms have just write API support, however, no read support, which requires us to fall back to Web scraping the website in order to retrieve data.
An example is Img.ly.

\begin{figure}
\centering
\includegraphics[width=1.0\linewidth,trim=16px 17px 12px 15px,clip]{./resources/davidsimonds.jpg}
\caption{David Simonds illustrates social networks as walled gardens due to their (by design) lock-in effects~\cite{DavidSimonds}.}
\label{fig:DavidSimonds}
\end{figure}

\section{Implementation Details}

\subsection{Data Structure}

\subsection{Media Item Collectors}

\subsection{Machine Translation}

\subsection{Part of Speech Tagging}

\subsection{Named Entity Disambiguation}

\subsection{Irrelevant Media Item Pruning}

\section{Experiments}
For our experiments, we have taken into account several events that happened on, or were still ongoing the morning of January 10, 2012, CET, and thus the subject of discussion on various social networks.
We have captured event-related media items and microposts, and made a subset of it available online\footnote{Event-related media items and microposts (subset): \todo{http://example.org/}}. 

\subsection{Considered Events}
In this Subsection, we give an objective and short overview on the context of the considered events in order to give the reader the necessary background knowledge.

\subsubsection{Assad Speech}
On January 10, 2012, Syrian President Bashar al-Assad delivered a lengthy televised talk strongly defending his government's actions and motivations, despite world pressure on his embattled government for its 10-month crackdown on protesters\footnote{Assad Speech: \url{http://www.cnn.com/2012/01/10/world/meast/syria-unrest/}}. 

\subsubsection{CES Las Vegas}
The International Consumer Electronics Show (CES) is a major technology-related trade show held each January in the Las Vegas Convention Center.
Not open to the public, the Consumer Electronics Association-sponsored show typically hosts previews of products and new product announcements\footnote{CES Las Vegas: \url{http://www.cesweb.org/aboutcea.asp}}.

\subsubsection{Christian Wulff Case}
Since December 2011, German President Christian Wulff faces controversy over discrepancies in statements about a loan while governor of Lower Saxony.
When the affair settled down, it was revealed that he had applied pressure on Springer Press to delay revelations on the issue until he was back from a visit abroad.
When Wulff found out that a tabloid was going to break the story, he left a message on the voice mail of the editor-in-chief in which he threatened to take legal action\footnote{Christian Wulff Case: \url{http://www.spiegel.de/international/germany/0,1518,804631,00.html}}.

\subsubsection{Cut the Rope Launch}
On January 10, 2012 during Microsoft's keynote at CES, the HTML5 version of the popular mobile game \textit{Cut the Rope} was announced\footnote{Cut the Rope Launch: \url{http://ces.cnet.com/8301-33377_1-57356403/}}.

\subsubsection{Dixville Notch}
Dixville Notch is an unincorporated village in Dixville township of Coos County, New Hampshire, USA, best known in connection with its longstanding middle-of-the-night vote in the U.S. presidential election.
In a tradition that started in the 1960 election, all the eligible voters in Dixville Notch gather at midnight in the ballroom of The Balsams.
This year, on January 10, 2012, the voters cast their ballots and the polls officially closed one minute later\footnote{Dixville Notch: \url{http://www.washingtonpost.com/2012/01/09/gIQANslKnP_story.html}}.

\subsubsection{Free Mobile Launch}
Free Mobile is a French mobile broadband company, part of the Iliad group.
On January 10, 2012, a long-awaited mobile phone package for €19.99 with calls included to 40 countries, texts, multimedia messages and Internet was announced by the Iliad group's Chief Strategy Officer, Xavier Niel\footnote{Free Mobile Launch: \url{http://www.nytimes.com/2012/01/11/technology/iliad-takes-aim-at-top-mobile-operators-in-france.html}}.

\subsubsection{Ubuntu TV Launch}
Ubuntu TV by Canonical, based on the user interface Unity, is a variant of the Ubuntu operating system, designed to be a Linux distribution specially adapted for embedded systems in televisions. It was announced by Canonical on January 10, 2012, at CES\footnote{Ubuntu TV: \url{http://www.theverge.com/2012/1/9/2695387/ubuntu-tv-video-hands-on}}.

\subsection{Discussion}

\section{Related Work} \label{sec:relatedwork}
\cite{Fabro2012}
Just Flickr and YouTube.
Use clustering algorithm for textual content.
Duplicate detection via Color and Edge Directivity Descriptor (CEDD).
Restrict selection to geolocated content and time period.

\cite{Liu2011}
Objective is to aggregate heterogeneous event information sources using Linked Data principles.
Just Flickr and YouTube, with additional media from event information sources.
Query with "what", "where", "when", uses "title"+"time" and "geotag"+"time".
Use machine-tagged media items to find visually similar media items.

\section{Future Work}
Privacy and licensing issues.
Support more social networks and improve Web scrapers.

\section{Conclusion}

\section{Acknowledgments}
This work was partially supported by the European Commission under Grant No. 248296 FP7 \mbox{I-SEARCH} project.

% back to normal size Computer Modern for URLs in bibliography
\let\ttdefault\oldttdefault
\let\url\oldurl

\bibliographystyle{abbrv}
\bibliography{icmr2012}

\balancecolumns
\end{document}